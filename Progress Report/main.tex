\documentclass{csfyp}
\usepackage{graphicx}

\title{The Malta Human Genome Project \\
  \large Progress Report}

\author{Sara Ann Abdilla (188396M)}
\supervisor{Jean Paul Ebejer}
\date{December 2016}

\begin{document}

\pagenumbering{roman} 
\tableofcontents
%\listoffigures
%\listoftables

\maketitle

\pagenumbering{arabic} 
\setcounter{page}{1}


\begin{abstract}
The abstract should act as a stand-alone (very) brief description of the whole story: The context, the solution, how effective it was found to be. There is no better way to learn how to write an abstract than by carefully reading the abstracts of good papers. This is usually the last part of the report to be written.
\end{abstract}


\section{Introduction}
\label{s:intro}

Over the years, more and more human reference genomes are being assembled.  A reference genome is a digital database consisting of DNA sequences, each individual's DNA corresponding to an arrangement of approximately 3 billion bases (made up from four nucleotide bases).  These possible bases are adenine, cytosine, guanine and thymine - oftenly referred to by the letters A, C, G and T respectively \cite{aiBk, introgenom}.  

Genome assembly/sequencing technologies are rapidly advancing and their costs are decreasing, both due to the fact that their importance is becoming more widely known.  DNA variations and mutations may correlate to diseases so discovering any approximately matching alignments between the reference genome and the DNA reads (sequencing reads) being analysed could very well help future medical diagnosis and treatments \cite{think1, think2, think3}.

The University of Malta is developing a National Maltese Human Reference Genome for this reason whereby whole genome sequencing on certain Maltese DNA samples (provided by the Malta BioBank) will be performed.    
   
An amount of challenges need to be tackled, some of which are listed hereunder:

\begin{enumerate}

  \item A substantial knowledge of bioinformatics - an area where computer technology is applied to biological data;

  \item A collections of known algorithms relating to DNA sequencing and their performance comparisons;

  \item Compression of reference genomes to improve efficiency, such studies having been already performed by Fritz et al \cite{refcompression} and Chen et al \cite{compression} among others;

  \item A data visualisation tool which promotes user-friendliness for even individuals who have no prior knowledge relating to the area in question;

  \item A genome browser which also promotes user-friendliness as stated previously.

\end{enumerate}


\section{Background and Literature Review}




\section{Aims and Objectives}

The aim of this project is to build tools for the visualization and analysis of genomes sequenced from the Malta Human Genome Project.  The research areas studied are bioinformatics, big data, data storage, data visualisation and data analysis among others.

The main objectives of the system are as follows:

\begin{enumerate}

  \item The alignment of a reference genome against a number of sequencing reads such as that done by Lee et al \cite{cgreads} and Li et al \cite{bwtransform} among others;

  \item The data visualiation of human genomes;

  \item The construction of a genome browser using novel and established components in order to reference DNA for comparitive genomics and to analyse DNA mutations.  

  \item A comparitive review of existing methods against all the above points.

\end{enumerate}


\section{Methods and Techniques Planned}



\section{Evaluation Stategy}

The research question which should be answered is


\section{Expected Deliverables}

\begin{enumerate}

  \item The Final Year Project (FYP) which will include: all the relevant background information required to understand said project, a detailed explanation of the system and the evaluation results; 

  \item The implementation of the designed and developed system;

  \item The documentation which will explain to the users how the system should be employed. 

\end{enumerate}


\section{References}

\bibliographystyle{alpha}
\nocite{*}
\bibliography{references}

\end{document}
